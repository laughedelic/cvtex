%------------------------------------
% Dario Taraborelli
% Typesetting your academic CV in LaTeX
%
% URL: http://nitens.org/taraborelli/cvtex
% DISCLAIMER: This template is provided for free and without any guarantee 
% that it will correctly compile on your system if you have a non-standard  
% configuration.
% Some rights reserved: http://creativecommons.org/licenses/by-sa/3.0/
%------------------------------------

%!TEX TS-program = xelatex
%!TEX encoding = UTF-8 Unicode

\documentclass[11pt, a4paper]{article}
\usepackage{fontspec} 

% DOCUMENT LAYOUT
\usepackage{geometry} 
\geometry{a4paper, textwidth=5.5in, textheight=8.5in, marginparwidth=.7in}
\setlength\parindent{0in}

% FONTS
\usepackage{xunicode}
\usepackage{xltxtra}
\defaultfontfeatures{Mapping=tex-text} % converts LaTeX specials (``quotes'' --- dashes etc.) to unicode
\setromanfont [Ligatures={Common}, BoldFont={Gentium Basic Bold}, ItalicFont={Gentium Basic Italic}]{Gentium Basic}
\setmonofont[Scale=0.8]{Monaco} 
% ---- CUSTOM AMPERSAND
%\newcommand{\amper}{{\fontspec[Scale=.95]{Gentium Basic Italic}\selectfont\itshape\&}}
% ---- MARGIN YEARS
\usepackage{marginnote}
\newcommand{\years}[1]{\marginnote{\footnotesize #1}}
\newcommand{\mnote}[1]{\years{\normalsize #1}}
\renewcommand*{\raggedleftmarginnote}{\raggedleft}
\setlength{\marginparsep}{10pt}
\reversemarginpar

% HEADINGS
\usepackage{sectsty} 
\usepackage[normalem]{ulem} 
\sectionfont{\rmfamily\mdseries\upshape\Large}
\subsectionfont{\rmfamily\bfseries\upshape\normalsize} 
\subsubsectionfont{\rmfamily\mdseries\upshape\normalsize} 

% PDF SETUP
\usepackage[ bookmarks, colorlinks, breaklinks
           , pdftitle={Alexey Alekhin CV}
           , pdfauthor={Alexey Alekhin}
           ]{hyperref}  
\hypersetup{linkcolor=blue,citecolor=blue,filecolor=black,urlcolor=blue} 

% DOCUMENT
\begin{document}

{\LARGE Alexey Alekhin}\\[1em]

\years{\normalsize Address:} Pyatiletok prosp. 17-1-8, \\
 Saint Petersburg, 193318, Russia\\[.2cm]
\years{\normalsize Born:}  June 26, 1989 — Lugansk, USSR\\[.2cm]
\years{\normalsize Phone:} \texttt{+7-911-1297576}\\
\years{\normalsize Email:} \href{mailto:alexey.alekhin@me.com}{alexey.alekhin@me.com}\\
\years{\normalsize Url:} \href{http://about.me/laughedelic}{http://about.me/laughedelic}
%\vfill

%%%%%%%%%%%%%%%%%%%%
\section*{Education}

\years{2006–2012}Saint Petersburg State University, Faculty of Mathematics and Mechanics, Specialist degree in Mathematics. \\
Diploma thesis: ``Cohomology of the 16 order modular group over the integer ring'';\\
\years{2003–2006}Saint Petersburg Laboratory for Continuous Mathematical Education (secondary education).

%%%%%%%%%%%%%%%%%%%%%%%%%%%%%%%
\section*{Additional education}

\years{July 2011}Leonhard Euler DAAD program (1.5 month) — Bielefeld University, Germany;\\
\years{2010–2011}Computer Science Club courses (1 semester) — Saint Petersburg Department of the Steklov Institute of Mathematics of the Russian Academy of Sciences;\\
\years{2010–2011}Bioinformatics courses (1 semester) — Saint Petersburg University of the Russian Academy of Sciences;\\
\years{1998–2003}Programming courses (5 years) — Computer Science Center of the Saint Petersburg Anichkov Youth Center.

%%%%%%%%%%%%%%%%%%%%%%%%%%%%%%%%%
\section*{Awards \& achievements}
%\subsection*{In the discipline of Mathematics}

\years{2006}Participation in the final of the {\it Intel International Science and Engineering Fair}, (Indianapolis, USA);\\
\years{2006}Second prize and the special award at the {\it Baltic Science and Engineering Fair}, (Saint Petersburg, Russia);\\
\years{2006}First prize at the {\it Junior Science and Engineering Fair}, (Moscow, Russia);\\
\years{2006}Second prize at the {\it XV Open Russian Scientific-Practical Conference}, (Moscow, Russia);\\
\years{2005}Third prize at the {\it XII International Conference of Young Scientists}, (Katowice, Poland);\\
\years{2001, 1999}First prize at the {\it local conference of Computer Science Center} of the Saint Petersburg Anichkov Youth Center.

%%%%%%%%%%%%%%%%%%%%%%%%%%%%%%%
\section*{Scientific interests}
 Homological algebra, Category theory, Functional programming, Lambda calculus.%, Mathematical logic

%%%%%%%%%%%%%%%%%%%%%%%%%%%%%%
\section*{Teaching experience}

\subsection*{Linear Algebra course}
\years{2011}Saint Petersburg State University (pedagogical practice, 1 semester)

\subsection*{``Basics of Functional Programming in Haskell'' course}
\years{2008–2011}Saint Petersburg Laboratory for Continuous Mathematical Education;\\
\years{2007–2009}Computer Science Center of the Saint Petersburg Anichkov Youth Center.

%%%%%%%%%%%%%%%%%%%%%%%%%%
\section*{Work experience}

\years{2011–present}Systema-Soft. Software development for the iOS mobile platform; user interface design (Objective C, Cocoa);\\
\years{2009–2010}Metrotek Science and Technology Center. Optimisation of algorithms for telecommunication measurements data analysis (C, Haskell).

%%%%%%%%%%%%%%%%%%%%%%%%%%
\section*{Computer skills}

\subsection*{Proficient user of}
Mac OS X, Linux, Windows operating systems; Maple computer algebra system; Vim editor; {\fontspec{Times New Roman}\TeX\ (\LaTeX/\XeTeX)} typesetting system.

\subsection*{Programming experience (in descending order)}
Haskell, Objective C, C.

%%%%%%%%%%%%%%%%%%%%
\section*{Languages}

{\it Russian:} native speaker;\\
{\it English:} IELTS band score 7.0 (June 2012).

\vfill{}

\begin{center}
{\scriptsize  Last updated: \today\- •\- 
% ---- PLEASE LEAVE THIS BACKLINK FOR ATTRIBUTION AS PER CC-LICENSE
Typeset in \href{http://nitens.org/taraborelli/cvtex}{
\fontspec{Times New Roman}\XeTeX }\\
% ---- FILL IN THE FULL URL TO YOUR CV HERE
%Source: \href{http://nitens.org/taraborelli/cvtex}{http://nitens.org/taraborelli/cvtex}
}
\end{center}

\end{document}
